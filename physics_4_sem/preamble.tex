\usepackage[top=15mm,left=12.5mm, right=12.5mm]{geometry}
\usepackage[T2A]{fontenc}
\usepackage[utf8]{inputenc}
\usepackage[english,russian]{babel}
\usepackage{url}
\usepackage{ucs} % Для \textdegree


\usepackage{amssymb, amsmath, multicol, amsthm}
\usepackage{mathrsfs}
\usepackage{graphicx}
\graphicspath{{images1/}}
\usepackage{hyperref}
\usepackage{titleps} % Настраивает верхний колонтитул
\usepackage{icomma} % "Умная" запятая: $0,2$ --- число, $0, 2$ --- перечисление
%\usepackage{wrapfig} % Обтекание рисунков текстом - очень НЕ рекомендую, можно использовать окружение minipage
\usepackage{xifthen} 
\usepackage{xspace}
\usepackage{esvect} %красивый значок вектора
\usepackage{multirow} %Позволяет набрать задачу с дано и чертой справа от него
\usepackage{lipsum}
\usepackage{wrapfig}
\usepackage{floatrow} 
\usepackage{enumitem}
\usepackage{rotpages}

\usepackage[rgb,table,xcdraw]{xcolor}
\hypersetup{				% Гиперссылки
	unicode=true,           % русские буквы в раздела PDF
	colorlinks=true,       	% false: ссылки в рамках; true: цветные ссылки
	linkcolor=black,          % внутренние ссылки
	citecolor=black,        % на библиографию
	filecolor=magenta,      % на файлы
	urlcolor=blue           % на URL
}

\usepackage{tikz}
\usetikzlibrary{arrows,decorations.pathmorphing,backgrounds,positioning,fit,petri}

\newpagestyle{main}{%
	\setheadrule{.4pt}%
	\sethead
	[\thepage][][\sectiontitle]
	{\subsectiontitle}{}{\thepage} }	
\pagestyle{main}

\let\leq\leqslant
\let\geq\geqslant

% Рисунки
\usepackage{graphicx}
\usepackage{wrapfig}
\usepackage[font=small,labelfont=bf]{caption}

\usepackage{hyperref}
\usepackage[rgb]{xcolor}
\hypersetup{				% Гиперссылки
    colorlinks=true,       	% false: ссылки в рамках
	urlcolor=blue          % на URL
}

%  Русский язык

\usepackage[T2A]{fontenc}			% кодировка
\usepackage[utf8]{inputenc}			% кодировка исходного текста
\usepackage[english,russian]{babel}	% локализация и переносы


% Математика
\usepackage{amsmath,amsfonts,amssymb,amsthm,mathtools} 
\usepackage{systeme}

\usepackage{wasysym}


%for inserting code
\usepackage{listings}
\usepackage{color}

\definecolor{dkgreen}{rgb}{0,0.6,0}
\definecolor{gray}{rgb}{0.5,0.5,0.5}
\definecolor{mauve}{rgb}{0.58,0,0.82}

\lstset{frame=tb,
  language=Java,
  aboveskip=3mm,
  belowskip=3mm,
  showstringspaces=false,
  columns=flexible,
  basicstyle={\small\ttfamily},
  numbers=none,
  numberstyle=\tiny\color{gray},
  keywordstyle=\color{blue},
  commentstyle=\color{dkgreen},
  stringstyle=\color{mauve},
  breaklines=true,
  breakatwhitespace=true,
  tabsize=3
}
